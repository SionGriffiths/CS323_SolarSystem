\documentclass[titlepage]{article}

\addtolength{\oddsidemargin}{-.875in}
\addtolength{\evensidemargin}{-.875in}
\addtolength{\textwidth}{1.75in}

\addtolength{\topmargin}{-.875in}
\addtolength{\textheight}{1.75in}


\usepackage[font=footnotesize]{caption}
\usepackage{verbatim}
\usepackage[section]{placeins}
\usepackage{graphicx}
\usepackage{hyperref}
\usepackage{tabularx} 



% --- XML Highlighting ---
\usepackage{listings}
\usepackage{color}
\definecolor{lightgray}{rgb}{.9,.9,.9}
\definecolor{darkgray}{rgb}{.4,.4,.4}
\definecolor{purple}{rgb}{0.65, 0.12, 0.82}
\definecolor{commentBlue}{rgb}{0.13, 0.53, 0.89}
\definecolor{stringGreen}{rgb}{0.43, 0.72, 0.26}
\definecolor{keywordOrange}{rgb}{0.97, 0.51, 0.15}

\lstdefinelanguage{JavaScript}{
  keywords={typeof, new, true, false, catch, function, return, null, catch, switch, var, if, in, while, do, else, case, break},
  keywordstyle=\color{blue}\bfseries,
  ndkeywords={class, export, boolean, throw, implements, import, this},
  ndkeywordstyle=\color{keywordOrange}\bfseries,
  identifierstyle=\color{black},
  sensitive=false,
  comment=[l]{//},
  morecomment=[s]{/*}{*/},
  commentstyle=\color{commentBlue}\ttfamily,
  stringstyle=\color{stringGreen}\ttfamily,
  morestring=[b]',
  morestring=[b]"
}

\lstset{
   language=JavaScript,
   backgroundcolor=\color{lightgray},
   extendedchars=true,
   basicstyle=\footnotesize\ttfamily,
   showstringspaces=false,
   showspaces=false,
   numbers=left,
   numberstyle=\footnotesize,
   numbersep=9pt,
   tabsize=2,
   breaklines=true,
   showtabs=false,
   captionpos=b
}
% --- ---- ---



% Title Page
\title{CS323 Assignment Report}
\author{Sion Griffiths - sig2}


% This just adds lines between paragraphs
\usepackage[parfill]{parskip}

%This allows diagrams to be added
\usepackage{graphicx}


\begin{document}
\maketitle
\tableofcontents




\newpage
\section{Introduction}

This report will detail the steps undertaken in constructing a visualisation of the Sun, Earth and Moon using WebGL and specifically the Three.js library. It should be noted that for this visualisation and throughout this document, the horizontal axes in the system are X and Z and the vertical is Y.

To access the visualisation use the following link -  \url{http://users.aber.ac.uk/sig2/cs323/CS323\_SolarSystem/}


\begin{figure}[h!]
                \centering
                \includegraphics[width=0.8\columnwidth]{media/overview1}
                \caption{A screen capture showing an overview of the developed system}
                \label{fig:basic_model}
            \end{figure}


\subsection{A note on image figures}
Unless otherwise stated, all images included in this document should be recreatable using the GUI options shown and careful manipulation of the camera using the standard orbit controls. If no options are shown or information given in the caption, default parameters should be assumed.
\section{Structure \& Implementation}
The code developed for this assignment can be split into 2 distinct packages, Model and Utils. The javascript files contained within the model package are essentially the descriptions of the entities within the system, the Earth, Sun and Moon. These files contain data regarding the state of the entities (geometry etc) along with functions which control their behaviour, for example the 'update' functions which control the changes to the entities(position etc) each iteration of the main logic loop.


The utils package contains utility code, split across various files depending on the nature of the code. The package contains files with utility code for the following: matrix functions, simple geometry functions and functions relating to the computation of orbits.


Separate from these two packages is the main.js file. This file serves as the entry point and initialisation of the system along with controlling the updating of the various parts of the system, entities, camera, GUI variables etc. 


The approach taken was an iterative one, starting with implementing the basic requirements using the native Three.js functionality, this allowed a basic understanding of the task at hand and provided a means to further familiarise myself with the Three.js framework. Once basic functionalities were implemented, prototyping of the extended requirements could begin, again in an iterative manner, with ongoing testing throughout. Testing was done mostly via careful observation and use of debugging tools native to browsers and the WebStorm IDE provided by JetBrains.

\subsection{Parameters}
The scales used in the visualisation are totally arbitrary, chosen to provide a visually pleasing and easy to navigate system rather than anything faithful to reality. Distances between objects with an elliptical orbit are the value for the semi-major axis length.
\subsubsection{Earth}
 \begin{tabularx}{0.75\columnwidth}{| X | X |}
            \hline
            Earth Radius & 4 units or 1 Earth radius \\
            \hline
            Earth - Sun distance  & 80 or 20 Earth radii \\
            \hline
            Earth Axial tilt & 23.4 degrees \\
            \hline
            Earth Axial rotation period & 1 day \\
            \hline
			Earth Orbit eccentricity & 0.12 \\
			\hline
			Earth Orbital period & 365.26 days \\
			\hline
        \end{tabularx} \\

\subsubsection{Moon}
 \begin{tabularx}{0.75\columnwidth}{| X | X |}
            \hline
            Moon Radius & 1 units or 0.25 Earth radii \\
            \hline
            Moon - Earth distance  & 10 or 2.5 Earth radii \\
            \hline            
            Moon Axial rotation period & 27.32 days \\
            \hline
			Moon Orbit eccentricity & 0.3 \\
			\hline
			Moon Orbital period & 27.32 days \\
			\hline
			Moon Orbital tilt & 5.145 degrees \\
			\hline
\end{tabularx} \\

\subsubsection{Sun}
 \begin{tabularx}{0.75\columnwidth}{| X | X |}
            \hline
            Sun Radius & 14 units or 3.5 Earth radii \\
            \hline
            Sun Axial rotation period & 27 days \\
            \hline			
        \end{tabularx} \\
\section{Orbits}

\subsection{Basics}
\subsubsection{Circular orbits}
\begin{figure}[h!]
                \centering
                \includegraphics[width=0.4\columnwidth]{media/cicOrb}
                \caption{Screen captures displaying the circular orbit of the Moon around the Earth, a code change is required to display a circular orbit within the finished system. This can be achieved by setting the first parameter of generateElliptical() to 0 on line 91 of main.js }
            \end{figure}
During the initial prototyping stage of development, the orbits of the Earth and Moon were implemented as simple circles. This was achieved using simple trigonometry to update the relative positions on the x and z axes. The function to achieve this is shown in the code snippet below,:

\begin{lstlisting}[caption=Simple orbit calculation]
var earthDistanceFromSun = 24;
var earthRotationSpeed = 0.2;

var eathRot = function(earthMesh, sunMesh){
    
    var x = earthDistanceFromSun * -Math.cos(earthRotationAngle * (Math.PI / 180));
    var z = earthDistanceFromSun * -Math.sin(earthRotationAngle * (Math.PI / 180));
    earthRotationAngle-= earthOrbitRotationSpeed;
    
    earthMesh.position.x = sunMesh.position.x + x;
    earthMesh.position.z = sunMesh.position.z + z;
}

\end{lstlisting}

As the value for \texttt{earthRotationAngle} decreases with each iteration of the function, the positions calculated will 'swing' around an origin point. The calculated positions are added onto the sun's central position and applied to the Earth's position, such that the Earth will appear to circle around the sun.

\subsubsection{The earth orbits the centre of the sun \& The moon orbits the centre of the earth}
Using the technique described above, it's clear to see that this can be applied to any mesh to ensure that it's rotation is around the centre of another. Even if the other mesh has a dynamic position (as, in the case of the Moon orbiting the Earth as the Earth is moving around the Sun.), as the positions applied are relative as opposed to being absolute. The relative positions can be added to any other position to create the desired circular orbit.
 

\subsection{Extended}
\subsubsection{Elliptical orbits \& Non-uniform orbital velocities}
Using the information given within the assignment brief, elliptical orbits and non-uniform orbital velocity functionality was implemented within the system. The provided equations, once translated into code, allowed an elliptical orbit to be pre-computed as a list of vectors that would then be traversed and applied as position for the centre of an orbiting mesh to provide an animated orbit.


\begin{figure}[h!]
                \centering
                \includegraphics[width=0.8\columnwidth]{media/ellipboth}
                \caption{Screen captures displaying the orbit of the Moon calculated with different eccentricity values. 0.5 on the left, 0.3 (system default) on the right. In order to change this value the code needs to be changed in main.js, line 91. The first parameter for the function generateElliptical()}
            \end{figure}
\FloatBarrier
The distance between the list of vectors generated for the orbit is not constant, the effect of this is a non-uniform orbital velocity as the points are traversed in the animation, giving an approximation of planetary motion.

The code for generating the elliptical is shown below:

\begin{lstlisting}[caption=Above code can be found in OrbitUtils.js and is adapted from the formulas given in the assignment brief.]
this.generateElliptical = function(eccentricity, periodSlices, semiMajorAxis, tiltValue){

    var orbitVerts = [];
    var theta = 0;
    var r = 0;

    while(theta <= 2*Math.PI) {
        theta += computeTheta(eccentricity,theta,periodSlices);
        r =  computeR(eccentricity,theta,semiMajorAxis);

        //negate the result for X to achieve anti clockwise orbital rotations
        var x = -polarXtoCart(r,theta);
        var z = polarZtoCart(r,theta);
        orbitVerts.push(vec3(x,0,z));
    }
    return orbitVerts;
};

//((2/N)( 1+ e cos ? )^2)/(1-e^2)^(3/2)
var computeTheta = function(e,theta,periodSlices){
    return ((2/periodSlices)* Math.pow((1+ (e*Math.cos(theta))),2))  / Math.pow( (1-Math.pow(e,2)) ,(3/2));
};

//r = a(1-e^2)/(1+ e cos ? )
var computeR = function(e, theta, a){
    return (a*(1-Math.pow(e,2)))/(1+e*Math.cos(theta));
};
\end{lstlisting}

Similar to the circular orbit described in an earlier section, the coordinate points are relative, they are generated with the scene origin (position 0,0,0) as the centre of their orbit. In the case of the Earth, since the centre of the Sun is at the origin point (0,0,0) no further action is required beyond specifying the orbit distance as the \texttt{semiMajorAxis} parameter in the above \texttt{generateElliptical} function. However, for the Moon to orbit the centre of the Earth we must apply these relative orbit positions to the current position of the Earth. The code to do so is detailed below:
\begin{lstlisting}[caption=Code taken from Moon.js showing the moon mesh position being set as that of the Earth plus the relative axis positions from the orbital point list]
this.getMesh().position.x = earth.getX() + this.orbitPoints[count].x;
this.getMesh().position.y = earth.getY() + this.orbitPoints[count].y;
this.getMesh().position.z = earth.getZ() + this.orbitPoints[count].z;

\end{lstlisting}


\subsubsection{Constant tilt of the Moon’s orbit}
\begin{figure}[h!]
                \centering
                \includegraphics[width=0.8\columnwidth]{media/moonTilt}
                \caption{Screen capture showing a trace of the Moon's orbit (gray) compared to the Earth's (pink)}
\end{figure}

The process of applying a constant tilt to the Moon's orbit is fairly straight forward. Given a list of vectors defining positional data for the orbit(as generated by the elliptical orbit code in the above section for example) and the desired angle of rotation, a matrix can be applied to each vector in the list in order to apply the desired transform.

In the case of the Moon's orbit we need to apply a rotation through a horizontal axis to achieve a tilt in the vertical. In this case the Z axis. The angle of rotation is passed into the below function and converted to radians before being applied as a standard z-rotation matrix. 
\begin{lstlisting}[caption=Above code can be found in OrbitUtils.js]
this.applyTiltToOrbit = function(angle, points){
    var tiltRotation =  getZRotationMatrixAsMat4(angle);
    for(var i = 0; i < points.length; i++){
        points[i] = applyMat4toVec3(tiltRotation,points[i]);
    }
    return points;
};
\end{lstlisting}

\begin{lstlisting}[caption=Definition of a rotation matrix in the Z axis. Can be found in MatrixUtils.js]
function getZRotationMatrixAsMat4 (angle){

    var radians = deg2rad(angle);
    var mat4 = new THREE.Matrix4();
    
    mat4.set(
        Math.cos(radians), -Math.sin(radians), 0, 0,
        Math.sin(radians), Math.cos(radians), 0, 0,
        0, 0, 1, 0,
        0, 0, 0, 1
    );
    
    return mat4;
}

\end{lstlisting}


\section{Textures and lighting}

\subsection{Basics}
\subsubsection{The sun, earth and moon shown as texture mapped spheres}

The spherical entities in the system (Sun, Earth and Moon) are meshes consisting of texture mapped sphere geometries. The geometries are defined as Three.js native SphereGeometry.

Texture mapping of the meshes was achieved using Three.js native loadTexture functionality, an example of the Earth mesh being defined is below:
\begin{lstlisting}[caption=Above code can be found in sun.js]
this.geometry = new THREE.SphereGeometry(4, 32, 32);
this.material = new THREE.MeshPhongMaterial();
this.material.map = THREE.ImageUtils.loadTexture('assets/images/earthmap1k.jpg');
\end{lstlisting}

\begin{figure}[h!]
                \centering
                \includegraphics[width=0.8\columnwidth]{media/textures}
                \caption{A screen capture showing texture detail}
\end{figure}


\subsubsection{The sun shown as a self-illuminated sphere}
In order for the Sun to appear as self illuminated, the material for its mesh has been defined as emissive. The sun texture image itself is used as an emission map and a colour parameter is given. 
\begin{lstlisting}
this.material =  new THREE.MeshPhongMaterial({
    emissive: 0xF2E9A6, //emission colour
    emissiveMap: new THREE.TextureLoader().load("assets/images/sunmap.jpg")
});
\end{lstlisting}
\FloatBarrier
\begin{figure}[h!]
                \centering
                \includegraphics[width=0.8\columnwidth]{media/sunShades}
                \caption{Emissive texture with different colours. The colour can be changed by using a different hex value on line 12 in sun.js}
\end{figure}
\FloatBarrier
\subsubsection{Earth and moon lit by a single point light source located at the centre of the Sun}
This is the only requirement which could be considered unfulfilled. Indeed there is no point light in the centre of the sun. However, the Earth and Moon are lit from a light source at the centre of the Sun, in order for shadows to be implemented it was necessary to use a source other than point light since it does not support such functionality. This is a limitation imposed by Three.js for efficiency sake. Please see section 4.2.2 below for details of the shadows.

\subsection{Extended}

\subsubsection{Lighting of the earth and moon in Phong shading}
Phong shading is implemented in the system for the Earth and the Moon via native Three.js functionalities. The desired shading is achieved through definition of mesh material as \texttt{THREE.MeshPhongMaterial()}, inbuilt Three.js functionality then correctly configures the shaders for these meshes if a compatible light is used, for example PointLight or SpotLight.

\begin{figure}[h!]
                \centering
                \includegraphics[width=0.8\columnwidth]{media/spec}
                \caption{Screen capture showing specular reflection details}
\end{figure}


\subsubsection{Non-illuminated parts of the earth and moon to be shown in shadow \& Eclipse shadows}
The shadows in the system are achieved via use of a SpotLight, located in the Sun. The SpotLight uses native Three.js functionality to cast shadows on meshes which have been explicitly enabled for shadows via configuration of the \texttt{mesh.castShadow} and \texttt{mesh.receiveShadow} parameters. The SpotLight gives the effect of radiating light from the sun in all directions but is in fact targeting the earth explicitly with each animation update. 


\begin{figure}[h!]
                \centering
                \includegraphics[width=0.8\columnwidth]{media/eclipse}
                \caption{Screen capture detailing the moon's eclipse shadow}
\end{figure}

\begin{figure}[h!]
                \centering
                \includegraphics[width=0.8\columnwidth]{media/shadow1}
                \caption{Screen capture showing shadow of non-illuminated surfaces}
\end{figure}
\medskip

\section{Axes and tilts}

\subsection{Basics}
\subsubsection{The sun, earth and moon each spinning on their own axes}
Using the GUI options to enable axes on the sun, earth and moon, it is clear to see all meshes rotating on their own axes. This could be achieved very simply with Three.js by using the \texttt{rotation.axes} functionality of a mesh and incrementing the value with each animation update. For the completed a different approach is taken and is detailed in a section below.
\begin{figure}[h!]
                \centering
                \includegraphics[width=0.8\columnwidth]{media/axes}
                \caption{Screen capture showing all axes}
\end{figure}

\subsection{Extended}
\subsubsection{Constant tilt of the Earth’s axis \& Earth's rotation}

\begin{figure}[h!]
                \centering
                \includegraphics[width=0.8\columnwidth]{media/earthTilt}
                \caption{Screen capture showing tilt of earth axes compared to the scene axes}
\end{figure}

The tilt of the Earth's axis is essentially defined by a matrix rotation in a horizontal axis. This is applied to the Earth mesh during initialisation using the Three.js \texttt{applyMatrix} functionality. The code to define an initial axial tilt of 23.4 degrees is shown below:
\begin{lstlisting}[caption=Code taken from Earth.js]
this.axialTiltMatrix = getZRotationMatrixAsMat4(23.4);
 this.mesh.applyMatrix(this.axialTiltMatrix);
\end{lstlisting}


The situation becomes slightly more complex when we need to define further matrix transforms on the same mesh, for instance in the case of the Earth, we'd like to rotate around the tilted axis. It is also desirable to be able to control the number of rotations completed in one orbital period such that a year in days can be simulated.
\\ 
In order to properly define a matrix to describe the axial rotation of the Earth we must calculate the amount of rotation per animation update. We can calculate the number of orbit points traversed in one 'day' with the following code, since the system will traverse one orbitPoint index per update :

\begin{lstlisting}[caption=Code taken from Earth.js]
 this.pointsInDay = (this.orbitPoints.length/365.26);
\end{lstlisting}
We can then use this \texttt{pointsInDay} value to calculate a rotation angle by simply dividing a whole rotation (360 degrees) by this value. We can then scale this value by the global simulation speed setting to keep everything synchronized as the user changes the speed. Once we have this value we can rotate the Earth mesh correctly per update. In order to rotate correctly and be able to preserve an axial tilt, the tilt matrix must be removed ( by applying the opposite rotation), the rotation then applied and the tilt reinstated. The code to do  so is detailed below:

\begin{lstlisting}[caption=Code taken from Earth.js]
//calculate rotation per update
rotValue = 360/this.pointsInDay*Math.floor(this.globalVars.simSpeed);
this.rotationMatrix = getYRotationMatrixAsMat4(rotValue);

var transToOrigin = new THREE.Matrix4().makeTranslation( -this.getX(), -this.getY(), -this.getZ());
//translate mesh to origin before applying transforms
this.mesh.applyMatrix(transToOrigin);

this.mesh.applyMatrix(this.removeAxialTiltMatrix);
this.mesh.applyMatrix(this.rotationMatrix);
this.mesh.applyMatrix(this.axialTiltMatrix);

//update position in pre-calculated orbit array
this.mesh.position.z = this.orbitPoints[count].z;
this.mesh.position.x = this.orbitPoints[count].x;

\end{lstlisting}

After the rotation, we can explicitly set the position of the mesh from the orbital points array without having to first translate from the origin since in this case the positional values in the orbital points are relative to the origin..

\subsubsection{Synchronous orbital and axial rotations of the moon}

Due to the nature of the Moon's motion when using an elliptical, non-constant velocity orbit it was decided to use a shortcut method to achieve the synchronisation of orbit and rotation: using the Three.js \texttt{lookAt()} function. Essentially this will set a constant face to the Earth for each animation update.

\section{User Interface}
\subsection{Suitable user control interface for viewing, scaling, timing}
A third party GUI framework, dat.GUI, was used to achieve the user controls for parameter settings along with using the OrbitControls.js bundled with Three.js to allow zooming and camera manipulations.

\begin{figure}[h!]
                \centering
                \includegraphics[width=0.4\columnwidth]{media/gui}
                \caption{Screen capture showing the GUI provided through dat.GUI}
\end{figure}

The implementation of the parameter controls was straight forward and well documented online. A list of variables to be controlled is defined with initial values:
\FloatBarrier
\begin{lstlisting}[caption=Code taken from main.js]
var guiVars = {
    ambientLightIntensity : 0.2,
    moonOrbitTrace : false,
    earthOrbitTrace : false,
    removeSun : false,
    paused : false,
    sceneAxes : false,
    earthAxes : false,
    moonAxes : false,
    sunAxes : false,
    shadowCam : false,
    simSpeed : 1
};


\end{lstlisting}

These are then added to a new dat.GUI object and logic based on their state can be implemented.
\section{Running the System}
As mentioned in the Introduction the visualisation can be accessed via \url{http://users.aber.ac.uk/sig2/cs323/CS323\_SolarSystem/}
For the local execution of the visualisation, unzip the files, change directory into the file CS323\_SolarSystem and open the index.html file with Firefox.  Currently other browsers may not function in this manner due to cross-site origin security functionality. In those cases please use the link provided.
\section{Credits}

\begin{itemize}
\item dat.GUI for the UI  - \url{https://code.google.com/p/dat-gui/}
\item Paul Bourke for the starmap - \url{http://paulbourke.net/}
\item planetpixelemporium for textures - \url{http://planetpixelemporium.com/}
\item learningthreejs for various tutorials - \url{http://learningthreejs.com/}
\end{itemize}

\section{Self Assessment}

 \begin{tabularx}{0.5\columnwidth}{| X | X | X |}
            \hline
            Basic Task & Documented & Implemented \\
            \hline
            1 & Yes & Yes \\
            \hline
            2 & Yes & Yes\\
            \hline
            3 & Yes & Yes \\
            \hline
            4 & Yes & Yes \\
            \hline
            5 & Yes & Yes \\
            \hline
            7 & Yes & Yes \\
            \hline
            7 & Yes & Yes \\
            \hline
        \end{tabularx} \\

        \vspace{1em}
        \begin{tabularx}{0.5\columnwidth}{| X | X | X |}
            \hline
            Extension & Documented & Implemented \\
            \hline
            a & Yes & Yes \\
            \hline
            b & Yes & Yes \\
            \hline
            c &Yes & Yes \\
            \hline
            d &Yes & Yes \\
            \hline
            e & Yes & Yes \\
            \hline
            f & Yes & Yes \\
            \hline
            g & Yes & Yes \\
            \hline
            h & Yes & Yes \\
            \hline
            i & Yes & Yes \\
            \hline
            j & Yes & Yes \\
            \hline
        \end{tabularx} \\

       
        \vspace{1em}
        \begin{tabularx}{0.5\columnwidth}{| X | c |}
            \hline
            Animation can be viewed in B57, C56 without additional software & Yes \\
            \hline
            All reference sources are acknowledged & Yes \\
            \hline
            Overall self-assessment out of 50 & 36 \\
            \hline
        \end{tabularx}




\end{document}