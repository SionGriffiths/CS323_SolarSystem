\documentclass[titlepage]{article}

\addtolength{\oddsidemargin}{-.875in}
\addtolength{\evensidemargin}{-.875in}
\addtolength{\textwidth}{1.75in}

\addtolength{\topmargin}{-.875in}
\addtolength{\textheight}{1.75in}


\usepackage[font=footnotesize]{caption}
\usepackage{verbatim}
\usepackage[section]{placeins}
\usepackage{graphicx}
\usepackage{hyperref}



% --- XML Highlighting ---
\usepackage{listings}

\usepackage{color}
\definecolor{gray}{rgb}{0.4,0.4,0.4}
\definecolor{darkblue}{rgb}{0.0,0.0,0.6}
\definecolor{cyan}{rgb}{0.0,0.6,0.6}

\lstset{
	basicstyle=\ttfamily,
	columns=fullflexible,
	showstringspaces=false,
	commentstyle=\color{gray}\upshape
}

\lstdefinelanguage{XML}
{
	morestring=[b]",
	morestring=[s]{>}{<},
	morecomment=[s]{<?}{?>},
	stringstyle=\color{black},
	identifierstyle=\color{darkblue},
	keywordstyle=\color{cyan},
	morekeywords={xmlns,version,type}% list your attributes here
}
% --- ---- ---



% Title Page
\title{CS323 Assignment Report}
\author{Sion Griffiths - sig2}


% This just adds lines between paragraphs
\usepackage[parfill]{parskip}

%This allows diagrams to be added
\usepackage{graphicx}


\begin{document}
\maketitle
\tableofcontents




\newpage
\section{Introduction}

This report will detail the steps undertaken in constructing a visualisation of the Sun, Earth and Moon using WebGL and specifically the Three.js library.

To access the visualisation use the following link -  \url{http://users.aber.ac.uk/sig2/cs323/CS323\_SolarSystem/}


\begin{figure}[h!]
                \centering
                \includegraphics[width=0.8\columnwidth]{media/overview1}
                \caption{A screen capture showing an overview of the developed system}
                \label{fig:basic_model}
            \end{figure}


\section{Structure \& Implementation}
The code developed for this assignment can be split into 2 distinct packages, Model and Utils. The javascript files contained within the model package are essentially the descriptions of the entities within the system, the Earth, Sun and Moon. These files contain data regarding the state of the entities (geometry etc) along with functions which control their behaviour, for example the 'update' functions which control the changes to the entities(position etc) each iteration of the main logic loop.


The utils package contains utility code, split across various files depending on the nature of the code. The package contains files with utility code for the following: matrix functions, simple geometry functions and functions relating to the computation of orbits.


Separate from these two packages is the main.js file. This file serves as the entry point and initialisation of the system along with controlling the updating of the various parts of the system, entities, camera, gui variables etc. 


The approach taken was an iterative one, starting with implementing the basic requirements using the native Three.js functionality, this allowed a basic understanding of the task at hand and provided a means to further familiarise myself with the Three.js framework. Once basic functionalities were implemented, prototyping of the extended requirements could begin, again in an iterative manner, since not everything worked as expected with Three.js to begin with.


\section{Basic Tasks}

\section{Extended Tasks}







 \begin{thebibliography}{1}
 	
 	\bibitem{replyBuy} David Uccelli, Multichannel Blog {\em http://www.multichannel-blog.co.uk/2012/02/13/80-of-portaltech-uk-hybris-partner-bought-for-1-6m/} Accessed September 2015
 	\bibitem{growthGraph} Reply LTD, Company profile {\em http://www.reply.eu/InvestorsDocuments/en/Company\_Profile\_eng.pdf} Accessed September 2015
 	\bibitem{stock} Google Finance, Reply S.p.A profile {\em https://www.google.co.uk/finance?q=BIT\%3AREY} Accessed September 2015
 	
 	
 	
 	
 		
 %	\bibitem{hoover} Dun \& Bradstreet, Hoover.com - Corporate Data Analyser and Aggregator {\em http://bit.ly/1Dj9e1m }

	


 \end{thebibliography}


\end{document}